\documentclass{beamer}
\usetheme{boxes}
\usepackage{qtree}
\renewcommand{\vec}[1]{\mathbf{#1}}
\definecolor{links}{HTML}{2A1B81}
\hypersetup{colorlinks,linkcolor=,urlcolor=links}

\title{An Empirical Evaluation of Explanations for State Repression}
\author[shortname]{Daniel W. Hill Jr. \inst{1} \and Zachary M. Jones \inst{2}}
\institute[shortinst]{\inst{1} University of Georgia \and %
  \inst{2} Penn State University}
\date{}

\begin{document}
\maketitle

\begin{frame}{The Empirical Literature on State Repression is Prolific}
  \begin{itemize}
  \item Goal is to discover political, economic, social conditions related to human rights abuse
  \item The most seminal study is Poe and Tate (1994)
    \begin{itemize}
    \item Democracy
    \item Civil/international war
    \item Economic development
    \item Population size
    \end{itemize}
  \item International economic factors
  \item Civil society/NGO activity
  \item Domestic legal institutions
  \item International law
  \end{itemize}
\end{frame}

\begin{frame}{Theoretical Claims Could be Better Evaluated}
  \begin{itemize}
  \item Importance of theoretically relevant indicators assessed by null hypothesis significance tests
  \item Statistical significance $\neq$ predictive power
  \item This does not prevent overfitting
  \end{itemize}
\end{frame}

\begin{frame}{What We Contribute}
  \begin{itemize}
  \item Cross-validation
    \begin{itemize}
    \item Does inclusion of the indicator improve out-of-sample fit? 
    \item Assessing out-of-sample fit guards against overfitting
    \end{itemize}
  \item Random forests
    \begin{itemize}
    \item No need to choose ``baseline'' model
    \item Allows for interactions/non-linear relationships
    \item Shows relative importance of variables
    \end{itemize}
  \end{itemize}
\end{frame}

\begin{frame}{Cross-Validation: What Is It?}
  \begin{itemize}
  \item A way to assess how good a model is at predicting the outcome of interest
  \item Split the data into $k$ folds, estimate a model on $k-1$ of the folds, and then predict the held out fold (saving the discrepancy statistic)
  \item Do this a lot of times to make sure that the discrepancy statistic that you got for a particular iteration isn't dependent on a particular split
  \item We use RMSE for OLS models and Somer's $D$ (a rank correlation coefficient) for the ordinal logit models
  \end{itemize}
\end{frame}

\begin{frame}{Cross-Validation: How Do We Do It?}
  \begin{itemize}
  \item Dependent variables: aggregated CIRI, PTS, CIRI sub-components
  \item Covariates: base model (two types) + variable of interest
  \item 7 dependent variables (sort-of), 62 model specifications, 10 folds, 1000 resampling iterations, on 5 imputed data sets = 21,700,000 models estimated (ack!)
  \end{itemize}
\end{frame}

\begin{frame}{CV Ordinal Logit: Political Imprisonment}
  \centering
  \includegraphics[width=\linewidth,height=.9\textheight,keepaspectratio]{./figures/cv-polpris.png}
\end{frame}

\begin{frame}{CV Ordinal Logit: Torture}
  \centering
  \includegraphics[width=\linewidth,height=.9\textheight,keepaspectratio]{./figures/cv-tort.png}
\end{frame}

\begin{frame}{CV OLS: Aggregated CIRI Index}
  \centering
  \includegraphics[width=\linewidth,height=.9\textheight,keepaspectratio]{./figures/cv-cwar-physint.png}
\end{frame}

\begin{frame}{CV OLS: Political Terror Scale}
  \centering
  \includegraphics[width=\linewidth,height=.9\textheight,keepaspectratio]{./figures/cv-cwar-pts-ols.png}
\end{frame}

\begin{frame}{Random Forests: Theory}
  \begin{itemize}
  \item Random forests are an ensemble of decision trees (we use a variation on the normal variety)
    \begin{itemize}
    \item Select a set of observations
    \item Select a set of variables
    \item Find which variable is the most strongly related to the outcome variable
    \item Find the split in the selected variable which optimally classifies observations
    \item Repeat until stopping criteria is met
    \item Repeat independently for each tree in the forest
    \end{itemize}
  \item Predicted class based on consensus vote from all trees in the forest
  \end{itemize}
\end{frame}

\begin{frame}{Random Forests: Variable Importance}
  \begin{itemize}
  \item If a variable is an important predictor, then randomly permuting its value should decrease classification/regression performance
  \item For each variable, test the null hypothesis that $P(Y, X_j, Z) = P(Y, Z)P(X_j)$ where $Y$ is the dependent variable, $X_j$ is the variable being permuted, and $Z$ are all $X_i, i \neq j$
  \end{itemize}
\end{frame}

\begin{frame}{Random Forests: Political Imprisonment}
  \centering
  \includegraphics[width=\linewidth,height=.9\textheight,keepaspectratio]{./figures/polpris-imp-sig.png}
\end{frame}

\begin{frame}{Random Forests: Torture}
  \centering
  \includegraphics[width=\linewidth,height=.9\textheight,keepaspectratio]{./figures/tort-imp-sig.png}
\end{frame}

\begin{frame}{What Do the Results Tell Us?}
  \begin{itemize}
  \item Civil war matters a lot
  \item Democracy matters a lot, but matters a lot more for some things
    \begin{itemize}
    \item Some Polity components measure repression
    \item This requires attention in the future
    \end{itemize}
  \item Concepts that receive far less attention matter a lot
    \begin{itemize}
    \item Domestic legal institutions 
      \begin{itemize}
      \item Judicial independence
      \item Common law
      \item Fair trial provisions
      \end{itemize}
    \item Oil rents
    \item Youth bulges
    \end{itemize}
  \end{itemize} 
\end{frame}

\begin{frame}{Thanks!}
  \begin{itemize}
  \item Data \& code online: \url{http://github.com/zmjones/eeesr}
  \item A draft of the paper (at \href{http://zmjones.com}{zmjones.com} or \href{http://myweb.fsu.edu/dwh06c}{myweb.fsu.edu/dwh06c}) and these slides are also online (on the PSS website or our personal sites)
  \item Email us comments if you have any: Danny (\href{mailto:dwhill@uga.edu}{dwhill@uga.edu}), Zach (\href{mailto:zmj@zmjones.com}{zmj@zmjones.com})
  \end{itemize}
\end{frame}

\begin{frame}{Random Forest Implementations}
  \begin{itemize}
  \item Decision trees are high variance estimators (hence the use of ensembles, aka random forests)
  \item Many implementations of decision trees/random forests are biased towards variables with more possible splits (\texttt{rpart}, \texttt{randomForest} in \texttt{R}, for example)
  \item Many implementations of random forests overfit (hence pruning after fitting)
  \item We use random forests with conditional inference trees (via \texttt{party}, in \texttt{R})
    \begin{itemize}
    \item Separates variable selection and variable splitting
    \item Uses linear statistics for selection and stopping rules 
    \item This prevents biased variable selection and overfitting
    \end{itemize}
  \end{itemize}
\end{frame}

\begin{frame}{Covariate Missingness}
  \begin{itemize}
  \item Multiple Imputation Using Chained Equations (MICE)
    \begin{itemize}
    \item Classification and Regression Trees for categorical data
    \item Random Indicator method for non-ignorably missing numeric data
    \end{itemize}
  \item Run cross-validation/random forests on each imputed data-set
  \item Pool results, and compute summary statistics
  \end{itemize}
\end{frame}

\begin{frame}{CV Ordinal Logit (with Civil War): Political Imprisonment}
  \centering
  \includegraphics[width=\linewidth,height=.9\textheight,keepaspectratio]{./figures/cv-cwar-polpris.png}
\end{frame}

\begin{frame}{CV Ordinal Logit: Disappearances}
  \centering
  \includegraphics[width=\linewidth,height=.9\textheight,keepaspectratio]{./figures/cv-disap.png}
\end{frame}

\begin{frame}{CV Ordinal Logit (with Civil War): Disappearances}
  \centering
  \includegraphics[width=\linewidth,height=.9\textheight,keepaspectratio]{./figures/cv-cwar-disap.png}
\end{frame}

\begin{frame}{CV Ordinal Logit: Killings}
  \centering
  \includegraphics[width=\linewidth,height=.9\textheight,keepaspectratio]{./figures/cv-kill.png}
\end{frame}

\begin{frame}{CV Ordinal Logit (with Civil War): Killings}
  \centering
  \includegraphics[width=\linewidth,height=.9\textheight,keepaspectratio]{./figures/cv-cwar-kill.png}
\end{frame}

\begin{frame}{CV Ordinal Logit: Political Terror Scale}
  \centering
  \includegraphics[width=\linewidth,height=.9\textheight,keepaspectratio]{./figures/cv-pts-lrm.png}
\end{frame}

\begin{frame}{CV Ordinal Logit (with Civil War): Political Terror Scale}
  \centering
  \includegraphics[width=\linewidth,height=.9\textheight,keepaspectratio]{./figures/cv-cwar-pts-lrm.png}
\end{frame}

\begin{frame}{CV OLS: Political Terror Scale}
  \centering
  \includegraphics[width=\linewidth,height=.9\textheight,keepaspectratio]{./figures/cv-pts-ols.png}
\end{frame}

\begin{frame}{CV OLS (with Civil War): Political Terror Scale}
  \centering
  \includegraphics[width=\linewidth,height=.9\textheight,keepaspectratio]{./figures/cv-cwar-pts-ols.png}
\end{frame}

\begin{frame}{CV OLS: Aggregated CIRI Index}
  \centering
  \includegraphics[width=\linewidth,height=.9\textheight,keepaspectratio]{./figures/cv-physint.png}
\end{frame}

\begin{frame}{CV OLS: Aggregated CIRI Index}
  \centering
  \includegraphics[width=\linewidth,height=.9\textheight,keepaspectratio]{./figures/cv-physint.png}
\end{frame}

\begin{frame}{Random Forests: Disapperances}
  \centering
  \includegraphics[width=\linewidth,height=.9\textheight,keepaspectratio]{./figures/disap-imp-sig.png}
\end{frame}

\begin{frame}{Random Forests: Killings}
  \centering
  \includegraphics[width=\linewidth,height=.9\textheight,keepaspectratio]{./figures/kill-imp-sig.png}
\end{frame}

\begin{frame}{Random Forests: Political Terror Scale}
  \centering
  \includegraphics[width=\linewidth,height=.9\textheight,keepaspectratio]{./figures/pts-imp-sig.png}
\end{frame}

\begin{frame}{Random Forests: Aggregated CIRI Index}
  \centering
  \includegraphics[width=\linewidth,height=.9\textheight,keepaspectratio]{./figures/physint-imp-sig.png}
\end{frame}

\end{document}